% Options for packages loaded elsewhere
\PassOptionsToPackage{unicode}{hyperref}
\PassOptionsToPackage{hyphens}{url}
%
\documentclass[
]{article}
\usepackage{amsmath,amssymb}
\usepackage{lmodern}
\usepackage{iftex}
\ifPDFTeX
  \usepackage[T1]{fontenc}
  \usepackage[utf8]{inputenc}
  \usepackage{textcomp} % provide euro and other symbols
\else % if luatex or xetex
  \usepackage{unicode-math}
  \defaultfontfeatures{Scale=MatchLowercase}
  \defaultfontfeatures[\rmfamily]{Ligatures=TeX,Scale=1}
\fi
% Use upquote if available, for straight quotes in verbatim environments
\IfFileExists{upquote.sty}{\usepackage{upquote}}{}
\IfFileExists{microtype.sty}{% use microtype if available
  \usepackage[]{microtype}
  \UseMicrotypeSet[protrusion]{basicmath} % disable protrusion for tt fonts
}{}
\makeatletter
\@ifundefined{KOMAClassName}{% if non-KOMA class
  \IfFileExists{parskip.sty}{%
    \usepackage{parskip}
  }{% else
    \setlength{\parindent}{0pt}
    \setlength{\parskip}{6pt plus 2pt minus 1pt}}
}{% if KOMA class
  \KOMAoptions{parskip=half}}
\makeatother
\usepackage{xcolor}
\usepackage[margin=1in]{geometry}
\usepackage{color}
\usepackage{fancyvrb}
\newcommand{\VerbBar}{|}
\newcommand{\VERB}{\Verb[commandchars=\\\{\}]}
\DefineVerbatimEnvironment{Highlighting}{Verbatim}{commandchars=\\\{\}}
% Add ',fontsize=\small' for more characters per line
\usepackage{framed}
\definecolor{shadecolor}{RGB}{248,248,248}
\newenvironment{Shaded}{\begin{snugshade}}{\end{snugshade}}
\newcommand{\AlertTok}[1]{\textcolor[rgb]{0.94,0.16,0.16}{#1}}
\newcommand{\AnnotationTok}[1]{\textcolor[rgb]{0.56,0.35,0.01}{\textbf{\textit{#1}}}}
\newcommand{\AttributeTok}[1]{\textcolor[rgb]{0.77,0.63,0.00}{#1}}
\newcommand{\BaseNTok}[1]{\textcolor[rgb]{0.00,0.00,0.81}{#1}}
\newcommand{\BuiltInTok}[1]{#1}
\newcommand{\CharTok}[1]{\textcolor[rgb]{0.31,0.60,0.02}{#1}}
\newcommand{\CommentTok}[1]{\textcolor[rgb]{0.56,0.35,0.01}{\textit{#1}}}
\newcommand{\CommentVarTok}[1]{\textcolor[rgb]{0.56,0.35,0.01}{\textbf{\textit{#1}}}}
\newcommand{\ConstantTok}[1]{\textcolor[rgb]{0.00,0.00,0.00}{#1}}
\newcommand{\ControlFlowTok}[1]{\textcolor[rgb]{0.13,0.29,0.53}{\textbf{#1}}}
\newcommand{\DataTypeTok}[1]{\textcolor[rgb]{0.13,0.29,0.53}{#1}}
\newcommand{\DecValTok}[1]{\textcolor[rgb]{0.00,0.00,0.81}{#1}}
\newcommand{\DocumentationTok}[1]{\textcolor[rgb]{0.56,0.35,0.01}{\textbf{\textit{#1}}}}
\newcommand{\ErrorTok}[1]{\textcolor[rgb]{0.64,0.00,0.00}{\textbf{#1}}}
\newcommand{\ExtensionTok}[1]{#1}
\newcommand{\FloatTok}[1]{\textcolor[rgb]{0.00,0.00,0.81}{#1}}
\newcommand{\FunctionTok}[1]{\textcolor[rgb]{0.00,0.00,0.00}{#1}}
\newcommand{\ImportTok}[1]{#1}
\newcommand{\InformationTok}[1]{\textcolor[rgb]{0.56,0.35,0.01}{\textbf{\textit{#1}}}}
\newcommand{\KeywordTok}[1]{\textcolor[rgb]{0.13,0.29,0.53}{\textbf{#1}}}
\newcommand{\NormalTok}[1]{#1}
\newcommand{\OperatorTok}[1]{\textcolor[rgb]{0.81,0.36,0.00}{\textbf{#1}}}
\newcommand{\OtherTok}[1]{\textcolor[rgb]{0.56,0.35,0.01}{#1}}
\newcommand{\PreprocessorTok}[1]{\textcolor[rgb]{0.56,0.35,0.01}{\textit{#1}}}
\newcommand{\RegionMarkerTok}[1]{#1}
\newcommand{\SpecialCharTok}[1]{\textcolor[rgb]{0.00,0.00,0.00}{#1}}
\newcommand{\SpecialStringTok}[1]{\textcolor[rgb]{0.31,0.60,0.02}{#1}}
\newcommand{\StringTok}[1]{\textcolor[rgb]{0.31,0.60,0.02}{#1}}
\newcommand{\VariableTok}[1]{\textcolor[rgb]{0.00,0.00,0.00}{#1}}
\newcommand{\VerbatimStringTok}[1]{\textcolor[rgb]{0.31,0.60,0.02}{#1}}
\newcommand{\WarningTok}[1]{\textcolor[rgb]{0.56,0.35,0.01}{\textbf{\textit{#1}}}}
\usepackage{graphicx}
\makeatletter
\def\maxwidth{\ifdim\Gin@nat@width>\linewidth\linewidth\else\Gin@nat@width\fi}
\def\maxheight{\ifdim\Gin@nat@height>\textheight\textheight\else\Gin@nat@height\fi}
\makeatother
% Scale images if necessary, so that they will not overflow the page
% margins by default, and it is still possible to overwrite the defaults
% using explicit options in \includegraphics[width, height, ...]{}
\setkeys{Gin}{width=\maxwidth,height=\maxheight,keepaspectratio}
% Set default figure placement to htbp
\makeatletter
\def\fps@figure{htbp}
\makeatother
\setlength{\emergencystretch}{3em} % prevent overfull lines
\providecommand{\tightlist}{%
  \setlength{\itemsep}{0pt}\setlength{\parskip}{0pt}}
\setcounter{secnumdepth}{5}
\ifLuaTeX
  \usepackage{selnolig}  % disable illegal ligatures
\fi
\IfFileExists{bookmark.sty}{\usepackage{bookmark}}{\usepackage{hyperref}}
\IfFileExists{xurl.sty}{\usepackage{xurl}}{} % add URL line breaks if available
\urlstyle{same} % disable monospaced font for URLs
\hypersetup{
  pdftitle={CRISPR Screen and Gene Expression Differential Analysis},
  pdfauthor={Lianbo Yu, Yue Zhao, Kevin R. Coombes, and Lang Li},
  hidelinks,
  pdfcreator={LaTeX via pandoc}}

\title{CRISPR Screen and Gene Expression Differential Analysis}
\author{Lianbo Yu, Yue Zhao, Kevin R. Coombes, and Lang Li}
\date{2022-07-27}

\begin{document}
\maketitle

{
\setcounter{tocdepth}{2}
\tableofcontents
}
\hypertarget{introduction}{%
\section{Introduction}\label{introduction}}

We developed CEDA to analyze read counts of single guide RNAs (sgRNAs)
raw CRISPR screening experiments. The sgRNAs are synthetically generated
from genes, and each gene can generate multiple sgRNAs. CEDA models the
sgRNA counts at different levels of gene expression by multi-component
normal mixtures, with the model fit by an EM algorithm. Posterior
estimates at sgRNA level are then summarized for each gene.

In this document, we use data from an experiment with the MDA231 cell
line to illustrate how to use CEDA to perform CRISPR screen data
analysis.

\hypertarget{overview}{%
\section{Overview}\label{overview}}

CEDA analysis follows a workflow that is typical for most omics level
experiments.

\begin{enumerate}
\def\labelenumi{\arabic{enumi}.}
\tightlist
\item
  Put the data into an appropriate format for input to CEDA.
\item
  Normalize the raw counts.
\item
  Fit a linear model to the data.
\item
  Summarize and view the results.
\end{enumerate}

\hypertarget{data-format}{%
\subsection{Data Format}\label{data-format}}

In our experiment, three samples of MDA231 cells were untreated at time
T=0, and another three samples of MDA231 cells were treated with DMSO at
time T=0. We are interested in detecting sgRNAs that are differentially
changed by a treatment.

The sgRNA read counts, along with a list of non-essential genes, are
stored in the dataset \texttt{mda231} that we have included in the
\texttt{CEDA} package. We read that dataset and explore its structure.

\begin{Shaded}
\begin{Highlighting}[]
\FunctionTok{library}\NormalTok{(CEDA)}
\FunctionTok{data}\NormalTok{(}\StringTok{"mda231"}\NormalTok{)}
\FunctionTok{class}\NormalTok{(mda231)}
\CommentTok{\#\textgreater{} [1] "list"}
\FunctionTok{length}\NormalTok{(mda231)}
\CommentTok{\#\textgreater{} [1] 2}
\FunctionTok{names}\NormalTok{(mda231)}
\CommentTok{\#\textgreater{} [1] "sgRNA"  "neGene"}
\end{Highlighting}
\end{Shaded}

As you can see, this is a list containing two components

\begin{enumerate}
\def\labelenumi{\arabic{enumi}.}
\tightlist
\item
  \texttt{sgRNA}, the observed count data of six samples, and
\item
  \texttt{neGene}, the set of non-essential genes.
\end{enumerate}

\begin{Shaded}
\begin{Highlighting}[]
\FunctionTok{dim}\NormalTok{(mda231}\SpecialCharTok{$}\NormalTok{sgRNA)}
\CommentTok{\#\textgreater{} [1] 23618     9}
\FunctionTok{length}\NormalTok{(mda231}\SpecialCharTok{$}\NormalTok{neGene}\SpecialCharTok{$}\NormalTok{Gene)}
\CommentTok{\#\textgreater{} [1] 200}
\FunctionTok{head}\NormalTok{(mda231}\SpecialCharTok{$}\NormalTok{sgRNA)}
\CommentTok{\#\textgreater{}                                   sgRNA  Gene DMSOa DMSOb DMSOc T0a T0b T0c}
\CommentTok{\#\textgreater{} 64780   chr19:10655652{-}10655671\_ATG4D\_{-} ATG4D   126   100   132  94  82  78}
\CommentTok{\#\textgreater{} 67381     chr5:32739109{-}32739128\_NPR3\_+  NPR3   266   452   309 557 687 587}
\CommentTok{\#\textgreater{} 67411    chr3:45515731{-}45515750\_LARS2\_+ LARS2    28    45    36 583 660 512}
\CommentTok{\#\textgreater{} 27053 chr12:111856039{-}111856058\_SH2B3\_+ SH2B3   509   501   661 578 824 636}
\CommentTok{\#\textgreater{} 55806   chr9:118163517{-}118163536\_DEC1\_+  DEC1   265   489   390 718 733 655}
\CommentTok{\#\textgreater{} 57274   chr1:228879268{-}228879287\_RHOU\_+  RHOU   144   124   137 160 164 119}
\CommentTok{\#\textgreater{}       exp.level.log2}
\CommentTok{\#\textgreater{} 64780    3.655866985}
\CommentTok{\#\textgreater{} 67381    0.200236907}
\CommentTok{\#\textgreater{} 67411    3.495375381}
\CommentTok{\#\textgreater{} 27053    4.227348316}
\CommentTok{\#\textgreater{} 55806    0.005794761}
\CommentTok{\#\textgreater{} 57274    0.925993344}
\end{Highlighting}
\end{Shaded}

Notice that the \texttt{sgRNA} component includes an extra column,
``\texttt{exp.level.log2}'', that are the expression level (in log2
scale) of genes and was computed from raw gene expression data.

The second element of the list \texttt{neGene} is, as expected, just a
list of gene names that are the non-essential genes:

\begin{Shaded}
\begin{Highlighting}[]
\FunctionTok{dim}\NormalTok{(mda231}\SpecialCharTok{$}\NormalTok{neGene)}
\CommentTok{\#\textgreater{} [1] 200   1}
\FunctionTok{head}\NormalTok{(mda231}\SpecialCharTok{$}\NormalTok{neGene)}
\CommentTok{\#\textgreater{}        Gene}
\CommentTok{\#\textgreater{} 189    MMD2}
\CommentTok{\#\textgreater{} 303    SUN5}
\CommentTok{\#\textgreater{} 155    KRT2}
\CommentTok{\#\textgreater{} 72    DEFA5}
\CommentTok{\#\textgreater{} 195   MUC17}
\CommentTok{\#\textgreater{} 70  CYP2C19}
\end{Highlighting}
\end{Shaded}

\hypertarget{normalization}{%
\subsection{Normalization}\label{normalization}}

The sgRNA read counts needs to be normalized across sample replicates
before formal analysis. The non-essential genes are assumed to have no
change after DMSO treatment. So, our recommended procedure is to perform
median normalization based on the set of non-essential genes.

\begin{Shaded}
\begin{Highlighting}[]
\NormalTok{mda231.ne }\OtherTok{\textless{}{-}}\NormalTok{ mda231}\SpecialCharTok{$}\NormalTok{sgRNA[mda231}\SpecialCharTok{$}\NormalTok{sgRNA}\SpecialCharTok{$}\NormalTok{Gene }\SpecialCharTok{\%in\%}\NormalTok{ mda231}\SpecialCharTok{$}\NormalTok{neGene}\SpecialCharTok{$}\NormalTok{Gene,]}
\NormalTok{cols }\OtherTok{\textless{}{-}} \FunctionTok{c}\NormalTok{(}\DecValTok{3}\SpecialCharTok{:}\DecValTok{8}\NormalTok{)}
\NormalTok{mda231.norm }\OtherTok{\textless{}{-}} \FunctionTok{medianNormalization}\NormalTok{(mda231}\SpecialCharTok{$}\NormalTok{sgRNA[,cols], mda231.ne[,cols])[[}\DecValTok{2}\NormalTok{]]}
\end{Highlighting}
\end{Shaded}

\hypertarget{analysis}{%
\subsection{Analysis}\label{analysis}}

Our primary goal is to detect essential sgRNAs that have different count
levels between conditions. We rely on the R package \texttt{limma} to
calculate log fold ratios between three untreated and three treated
samples.

\hypertarget{calculating-fold-ratios}{%
\subsubsection{Calculating fold ratios}\label{calculating-fold-ratios}}

First, we have to go through the usual \texttt{limma} steps to describe
the design of the study. There were two groups of replicate samples. We
will call these groups ``Control'' and ``Baseline'' (although
``Treated'' and Untreated'' would work just as well). Our main interest
is determining the differences between the groups. And we have to record
this information in a ``contrast matrix'' so limma knows what we want to
compare.

\begin{Shaded}
\begin{Highlighting}[]
\NormalTok{group }\OtherTok{\textless{}{-}} \FunctionTok{gl}\NormalTok{(}\DecValTok{2}\NormalTok{, }\DecValTok{3}\NormalTok{, }\AttributeTok{labels=}\FunctionTok{c}\NormalTok{(}\StringTok{"Control"}\NormalTok{,}\StringTok{"Baseline"}\NormalTok{))}
\NormalTok{design }\OtherTok{\textless{}{-}} \FunctionTok{model.matrix}\NormalTok{(}\SpecialCharTok{\textasciitilde{}}  \DecValTok{0} \SpecialCharTok{+}\NormalTok{ group)}
\FunctionTok{colnames}\NormalTok{(design) }\OtherTok{\textless{}{-}} \FunctionTok{sapply}\NormalTok{(}\FunctionTok{colnames}\NormalTok{(design), }\ControlFlowTok{function}\NormalTok{(x) }\FunctionTok{substr}\NormalTok{(x, }\DecValTok{6}\NormalTok{, }\FunctionTok{nchar}\NormalTok{(x)))}
\NormalTok{contrast.matrix }\OtherTok{\textless{}{-}} \FunctionTok{makeContrasts}\NormalTok{(}\StringTok{"Control{-}Baseline"}\NormalTok{, }\AttributeTok{levels=}\NormalTok{design)}
\end{Highlighting}
\end{Shaded}

Finally, we can run the lmima algorithm.

\begin{Shaded}
\begin{Highlighting}[]
\NormalTok{limma.fit }\OtherTok{\textless{}{-}} \FunctionTok{runLimma}\NormalTok{(}\FunctionTok{log2}\NormalTok{(mda231.norm}\SpecialCharTok{+}\DecValTok{1}\NormalTok{),design,contrast.matrix)}
\end{Highlighting}
\end{Shaded}

We merge the results from our limma analysis with the original sgRNA
count data.

\begin{Shaded}
\begin{Highlighting}[]
\NormalTok{mda231.limma }\OtherTok{\textless{}{-}} \FunctionTok{data.frame}\NormalTok{(mda231}\SpecialCharTok{$}\NormalTok{sgRNA, limma.fit)}
\FunctionTok{head}\NormalTok{(mda231.limma)}
\CommentTok{\#\textgreater{}                                   sgRNA  Gene DMSOa DMSOb DMSOc T0a T0b T0c}
\CommentTok{\#\textgreater{} 64780   chr19:10655652{-}10655671\_ATG4D\_{-} ATG4D   126   100   132  94  82  78}
\CommentTok{\#\textgreater{} 67381     chr5:32739109{-}32739128\_NPR3\_+  NPR3   266   452   309 557 687 587}
\CommentTok{\#\textgreater{} 67411    chr3:45515731{-}45515750\_LARS2\_+ LARS2    28    45    36 583 660 512}
\CommentTok{\#\textgreater{} 27053 chr12:111856039{-}111856058\_SH2B3\_+ SH2B3   509   501   661 578 824 636}
\CommentTok{\#\textgreater{} 55806   chr9:118163517{-}118163536\_DEC1\_+  DEC1   265   489   390 718 733 655}
\CommentTok{\#\textgreater{} 57274   chr1:228879268{-}228879287\_RHOU\_+  RHOU   144   124   137 160 164 119}
\CommentTok{\#\textgreater{}       exp.level.log2         lfc        se            p}
\CommentTok{\#\textgreater{} 64780    3.655866985  0.52954908 0.1093812 6.287987e{-}03}
\CommentTok{\#\textgreater{} 67381    0.200236907 {-}0.81763714 0.3240186 2.269143e{-}02}
\CommentTok{\#\textgreater{} 67411    3.495375381 {-}3.94431072 0.2746740 1.624301e{-}06}
\CommentTok{\#\textgreater{} 27053    4.227348316 {-}0.23583315 0.1257939 1.437600e{-}01}
\CommentTok{\#\textgreater{} 55806    0.005794761 {-}0.87573339 0.3301200 1.830149e{-}02}
\CommentTok{\#\textgreater{} 57274    0.925993344 {-}0.07125132 0.1276733 6.343237e{-}01}
\end{Highlighting}
\end{Shaded}

\hypertarget{fold-ratios-under-the-null-hypotheses}{%
\subsubsection{Fold ratios under the null
hypotheses}\label{fold-ratios-under-the-null-hypotheses}}

Under the null hypothses, all sgRNAs levels are unchanged between the
two conditions. To obtain fold ratios under the null, samples were
permuted between two conditions, and log fold ratios were obtained from
limma analysis under each permutation.

\begin{Shaded}
\begin{Highlighting}[]
\NormalTok{betanull }\OtherTok{\textless{}{-}} \FunctionTok{permuteLimma}\NormalTok{(}\FunctionTok{log2}\NormalTok{(mda231.norm }\SpecialCharTok{+} \DecValTok{1}\NormalTok{), design, contrast.matrix, }\DecValTok{20}\NormalTok{)}
\NormalTok{theta0 }\OtherTok{\textless{}{-}} \FunctionTok{sd}\NormalTok{(betanull)}
\NormalTok{theta0}
\CommentTok{\#\textgreater{} [1] 0.4601033}
\end{Highlighting}
\end{Shaded}

\hypertarget{fitting-three-component-mixture-models}{%
\subsubsection{Fitting three-component mixture
models}\label{fitting-three-component-mixture-models}}

A three-component mixture model (unchanged, overexpresssed, and
underexpressed) is assumed for log fold ratios at different level of
gene expression. Empirical Bayes method was employed to estimate
parematers of the mixtures and posterior means were obtained for
estimating actual log fold ratios between the two conditions. P-values
of sgRNAs were then calculated by permutation method.

\begin{Shaded}
\begin{Highlighting}[]
\NormalTok{nmm.fit }\OtherTok{\textless{}{-}} \FunctionTok{normalMM}\NormalTok{(mda231.limma, theta0)}
\end{Highlighting}
\end{Shaded}

Results from the mixture model were shown in Figure \(1\). False
discovery rate of \(0.05\) was used for declaring significant changes in
red color between the two conditions for sgRNAs.

\begin{Shaded}
\begin{Highlighting}[]
\FunctionTok{scatterPlot}\NormalTok{(nmm.fit}\SpecialCharTok{$}\NormalTok{data,}\AttributeTok{fdr=}\FloatTok{0.05}\NormalTok{,}\AttributeTok{xlim=}\FunctionTok{c}\NormalTok{(}\SpecialCharTok{{-}}\FloatTok{0.5}\NormalTok{,}\DecValTok{12}\NormalTok{),}\AttributeTok{ylim=}\FunctionTok{c}\NormalTok{(}\SpecialCharTok{{-}}\DecValTok{8}\NormalTok{,}\DecValTok{5}\NormalTok{))}
\end{Highlighting}
\end{Shaded}

\begin{figure}
\centering
\includegraphics{P:/project/Research/CRISPR/CRAN/CEDA/vignettes/Userguide_files/figure-latex/fig1-1.pdf}
\caption{Log fold ratios of sgRNAs vs.~gene expression level}
\end{figure}

\hypertarget{gene-level-summarization}{%
\subsubsection{Gene level
summarization}\label{gene-level-summarization}}

From the p-values of sgRNAs, gene level p-values were obtained by using
modified robust rank aggregation method (alpha-RRA). Log fold ratios
were also summarized at gene level.

\begin{Shaded}
\begin{Highlighting}[]
\NormalTok{mda231.nmm }\OtherTok{\textless{}{-}}\NormalTok{ nmm.fit[[}\DecValTok{1}\NormalTok{]]}
\NormalTok{p.gene }\OtherTok{\textless{}{-}} \FunctionTok{calculateGenePval}\NormalTok{(}\FunctionTok{exp}\NormalTok{(mda231.nmm}\SpecialCharTok{$}\NormalTok{log\_p), mda231.nmm}\SpecialCharTok{$}\NormalTok{Gene, }\FloatTok{0.05}\NormalTok{)}
\NormalTok{fdr.gene }\OtherTok{\textless{}{-}}\NormalTok{ stats}\SpecialCharTok{::}\FunctionTok{p.adjust}\NormalTok{(p.gene}\SpecialCharTok{$}\NormalTok{pvalue, }\AttributeTok{method =} \StringTok{"fdr"}\NormalTok{)}
\NormalTok{lfc.gene }\OtherTok{\textless{}{-}} \FunctionTok{calculateGeneLFC}\NormalTok{(mda231.nmm}\SpecialCharTok{$}\NormalTok{lfc, mda231.nmm}\SpecialCharTok{$}\NormalTok{Gene)}
\end{Highlighting}
\end{Shaded}


\end{document}
